\newcommand{\cvut}{České vysoké učení technické v~Praze}
\newcommand{\fjfi}{Fakulta jaderná a fyzikálně inženýrská}
\newcommand{\km}{Katedra matematiky}
\newcommand{\obor}{Inženýrská Informatika}
\newcommand{\zamereni}{Softwarové inženýrství a matematická informatika}

\newcommand{\nazevcz}{Výpočetní prostředí pro asimilaci disperzních atmosférických modelů}
\newcommand{\nazeven}{Environment for Assimilation of Atmospheric Dispersion Models}
\newcommand{\autor}{Matěj Laitl}
\newcommand{\rok}{2014}
\newcommand{\vedouci}{Ing. Václav Šmídl, Ph.D.}

\newcommand{\pracovisteVed}{Oddělení adaptivních systémů \\
	Ústav teorie informace a automatizace \\
	Akademie věd České republiky}
\newcommand{\konzultant}{}
\newcommand{\pracovisteKonz}{}

\newcommand{\klicova}{asimilace, disperzní model, sekvenční Monte Carlo, radioaktivní únik,
softwarová analýza, numerické výpočty, Python, Cython}
\newcommand{\keyword}{assimilation, dispersion model, sequential Monte Carlo, radioactive
release, software analysis, numerical computing, Python, Cython}
% Abstrakt práce: (cca 7 vět, min. 80 slov)
\newcommand{\abstrCZ}{Tato práce je zaměřena na návrh a implementaci softwarového řešení pro 
asimilaci atmosferických disperzních modelů, které simulují únik radioaktivních látek. Důraz je 
kladen na výpočetní efektivitu řešení. K asimilaci je přistupováno pomocí sekvenčního Monte Carlo 
vzorkování, které je v textu speciálně upraveno pro daný problém. Softwarová analýza vyústí ve 
vytvoření projektu Asim, který implementuje asimilaci, vylepšení knihovny pro Bayesovskou filtraci 
PyBayes a vytvoření knihovny pro lineární algebru Ceygen, která je použita ke zvýšení výkonu 
použitého implementačního prostředí --- Pythonu, který je kompilován projektem Cython. Je ukázáno, 
že Ceygen přináší 2- až 10-násobný nárůst výkonu algebraických metod pro určité velikosti vstupů. 
Nakonec je ověřena správná funkčnost celého asimilačního řešení pomocí dvojného experimentu.}
\newcommand{\abstrEN}{This text is devoted to designing and implementing software solution for 
assimilation of atmospheric dispersion models that simulate radioactive pollutant release, with 
accent of computational efficiency. The assimilation is approached using sequential Monte Carlo 
methods, which are tailored to this specific problem in the text. Software analysis results in 
creation of the Asim project to perform the assimilation, improvements to the PyBayes Bayesian 
filtering library and creation of the Ceygen linear algebra library that is used to accelerate 
computations in the implementation environment of choice, Cython-compiled Python. Ceygen is later 
shown to provide \(2\times\) to \(10\times\) performance increase of algebraic methods for certain 
problem sizes. Functionality of the assimilation solution is successfully verified using a twin 
experiment.}

%%% zde zacina kresleni dokumentu

% titulní strana
\thispagestyle{empty}

\begin{center}
	{\Large  \bf  \cvut\\[2mm] \fjfi }
	\vspace{10mm}

	\begin{tabular}{c}
	{\bf \km}\\
	{\bf Obor: \obor}\\
	{\bf Zaměření: \zamereni}
	\end{tabular}

	\vspace{10mm} \epsfysize=20mm  \epsffile{cvut-logo-bw-600} \vspace{15mm}

	{\LARGE
	\textbf{\nazevcz}
	\par}

	\vspace{5mm}

	{\LARGE
	\textbf{\nazeven}
	\par}

	\vspace{30mm}
	{\Large DIPLOMOVÁ PRÁCE}

\end{center}

\vfill
{\large
\begin{tabular}{rl}
Vypracoval: & \autor\\
Vedoucí práce: & \vedouci\\
Rok: & \rok
\end{tabular}
}

% zadání bakalářské práce
\newpage
\thispagestyle{empty} Před svázáním místo téhle stránky \fbox{vložíte zadání práce} s podpisem
děkana!!!!

% prohlášení
\newpage
\thispagestyle{empty}
~
\vfill


{\noindent}{\LARGE Prohlášení}

\vspace{0.5cm}
Prohlašuji, že jsem předloženou práci vypracoval samostatně a že jsem uvedl veškerou použitou
literaturu.

\vspace{5mm}V Praze dne ....................\hfill
    \begin{tabular}{c}
    ........................................\\
    \autor
    \end{tabular}

% poděkování
\newpage
\thispagestyle{empty}
~
\vfill

{\noindent}{\LARGE Poděkování}

\vspace{5mm}
Děkuji Ing.\ Václavu Šmídlovi, Ph.D. za vedení mé práce, cenné konzultace a houževnatost, s níž
mne motivoval posunovat práci kupředu. Dále děkuji doc.\ Ing.\ Jiřímu Fürstovi, Ph.D., jehož 
přednáška Numerický software mě inspirovala k vytvoření knihovny Ceygen.

\begin{flushright}
\autor
\end{flushright}

% strana s abstraktem
\newpage
\thispagestyle{empty}

\newbox\odstavecbox
\newlength\vyskaodstavce
\newcommand\odstavec[2]{%
    \setbox\odstavecbox=\hbox{%
         \parbox[t]{#1}{#2\vrule width 0pt depth 4pt}}%
    \global\vyskaodstavce=\dp\odstavecbox
    \box\odstavecbox}
\newcommand{\delka}{120mm}

\noindent\begin{tabular}{@{}ll}
  {\em Název práce:} & ~ \\
  \multicolumn{2}{@{}l}{\odstavec{\textwidth}{\bf \nazevcz}} \\[5mm]
  {\em Autor:} & \autor \\[5mm]
  {\em Obor:} & \obor \\
  {\em Druh práce:} & Diplomová práce \\[5mm]
  {\em Vedoucí práce:} & \odstavec{\delka}{\vedouci \\ \pracovisteVed} \\[5mm]
  \multicolumn{2}{@{}l}{\odstavec{\textwidth}{{\em Abstrakt:} ~ \abstrCZ \\ }} \\[5mm]
  {\em Klíčová slova:} & \odstavec{\delka}{\klicova} \\[10mm]

  {\em Title:} & ~\\
  \multicolumn{2}{@{}l}{\odstavec{\textwidth}{\bf \nazeven}}\\[5mm]
  {\em Author:} & \autor \\[5mm]
  \multicolumn{2}{@{}l}{\odstavec{\textwidth}{{\em Abstract:} ~ \abstrEN \\ }} \\[5mm]
  {\em Key words:} & \odstavec{\delka}{\keyword}
\end{tabular}
