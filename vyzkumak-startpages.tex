\newcommand{\cvut}{České vysoké učení technické v~Praze}
\newcommand{\fjfi}{Fakulta jaderná a fyzikálně inženýrská}
\newcommand{\km}{Katedra matematiky}
\newcommand{\obor}{Inženýrská Informatika}
\newcommand{\zamereni}{Softwarové inženýrství a matematická informatika}

\newcommand{\nazevcz}{Sekvenční Monte Carlo metody pro asimilaci disperzního atmosférického modelu}
\newcommand{\nazeven}{Sequential Monte Carlo Methods for Atmospheric Dispersion Model Assimilation}
\newcommand{\autor}{Matěj Laitl}
\newcommand{\rok}{2012}
\newcommand{\vedouci}{Ing. Václav Šmídl, Ph.D.}

\newcommand{\pracovisteVed}{Oddělení adaptivních systémů \\
	Ústav teorie informace a automatizace \\
	Akademie věd České republiky}
\newcommand{\konzultant}{Ing. Radek Hofman, Ph.D.}
\newcommand{\pracovisteKonz}{Oddělení adaptivních systémů \\
	Ústav teorie informace a automatizace \\
	Akademie věd České republiky}

\newcommand{\klicova}{asimilace, Bayesovská filtrace, sekvenční Monte Carlo, radioaktivní únik}
\newcommand{\keyword}{assimilation, Bayesian estimation, sequential Monte Carlo, radioactive
release}
% Abstrakt práce: (cca 7 vět, min. 80 slov)
\newcommand{\abstrCZ}{Práce pojednává o asimilaci atmosférického radioaktivního úniku pomocí metod
sekvenčního Monte Carlo vzorkování, což je relativně neprozkoumaná aplikace známé metodiky.
Nejprve jsou prezentovány zmíněné metody a jejich aplikace na daný problém, následně se práce
zaměřuje na volbu tzv. ``proposal'' funkce MC metody. Naivní a tzv. sdružená proposal funkce
jsou následně porovnány, z čehož technika sdruženého proposalu vychází jako jasný vítěz, i když se
uvažuje složitost, kterou do algoritmu vnáší. Práce je zakončena rozebráním několika
implementačních detailů, z nichž nejzajímavější je 184-násobné urychlení Python kódu nástrojem
Cython.}
\newcommand{\abstrEN}{The text deals with assimilation of an atmospheric radioactive release using
the sequential Monte Carlo technique, which is a rather unexplored application of the well-known
framework. The methods and their application to the given problem is presented and then the stress
it put on the choice of the proposal density of the particle filter. Naive and conjugate proposals
are then compared side-by-side, with conjugate proposal being shown as superior even accounting for
its relative complexity. Implementation details are discussed and \(\unit[184]{\times}\) speed-up
of the Python code is achieved using Cython.}

%%% zde zacina kresleni dokumentu

% titulní strana
\thispagestyle{empty}

\begin{center}
	{\Large  \bf  \cvut\\[2mm] \fjfi }
	\vspace{10mm}

	\begin{tabular}{c}
	{\bf \km}\\
	{\bf Obor: \obor}\\
	{\bf Zaměření: \zamereni}
	\end{tabular}

	\vspace{10mm} \epsfysize=20mm  \epsffile{cvut-logo-bw-600} \vspace{15mm}

	{\LARGE
	\textbf{\nazevcz}
	\par}

	\vspace{5mm}

	{\LARGE
	\textbf{\nazeven}
	\par}

	\vspace{30mm}
	{\Large VÝZKUMNÝ ÚKOL}

\end{center}

\vfill
{\large
\begin{tabular}{rl}
Vypracoval: & \autor\\
Vedoucí práce: & \vedouci\\
Rok: & \rok
\end{tabular}
}

% zadání bakalářské práce
\newpage
\thispagestyle{empty} Před svázáním místo téhle stránky \fbox{vložíte zadání práce} s podpisem
děkana!!!!

% prohlášení
\newpage
\thispagestyle{empty}
~
\vfill


{\noindent}{\LARGE Prohlášení}

\vspace{0.5cm}
Prohlašuji, že jsem předloženou práci vypracoval samostatně a že jsem uvedl veškerou použitou
literaturu.

\vspace{5mm}V Praze dne ....................\hfill
    \begin{tabular}{c}
    ........................................\\
    \autor
    \end{tabular}

% poděkování
\newpage
\thispagestyle{empty}
~
\vfill

{\noindent}{\LARGE Poděkování}

\vspace{5mm}
Děkuji Ing.\ Václavu Šmídlovi, Ph.D. za vedení mé práce, četné konzultace a trpělivé vysvětlování
matematických struktur problému, které vedlo k lepšímu návrhu softwarové implementace. Dále děkuji
Ing.\ Radku Hofmanovi, Ph.D. za konzultace a ukázku implementace puff modelu, která dovolila ověřit
správnost vlastního programu.

\begin{flushright}
\autor
\end{flushright}

% strana s abstraktem
\newpage
\thispagestyle{empty}

\newbox\odstavecbox
\newlength\vyskaodstavce
\newcommand\odstavec[2]{%
    \setbox\odstavecbox=\hbox{%
         \parbox[t]{#1}{#2\vrule width 0pt depth 4pt}}%
    \global\vyskaodstavce=\dp\odstavecbox
    \box\odstavecbox}
\newcommand{\delka}{120mm}

\noindent\begin{tabular}{@{}ll}
  {\em Název práce:} & ~ \\
  \multicolumn{2}{@{}l}{\odstavec{\textwidth}{\bf \nazevcz}} \\[5mm]
  {\em Autor:} & \autor \\[5mm]
  {\em Obor:} & \obor \\
  {\em Druh práce:} & Výzkumný úkol \\[5mm]
  {\em Vedoucí práce:} & \odstavec{\delka}{\vedouci \\ \pracovisteVed} \\[5mm]
  {\em Konzultant:} & \odstavec{\delka}{\konzultant \\ \pracovisteKonz} \\[5mm]
  \multicolumn{2}{@{}l}{\odstavec{\textwidth}{{\em Abstrakt:} ~ \abstrCZ \\ }} \\[5mm]
  {\em Klíčová slova:} & \odstavec{\delka}{\klicova} \\[10mm]

  {\em Title:} & ~\\
  \multicolumn{2}{@{}l}{\odstavec{\textwidth}{\bf \nazeven}}\\[5mm]
  {\em Author:} & \autor \\[5mm]
  \multicolumn{2}{@{}l}{\odstavec{\textwidth}{{\em Abstract:} ~ \abstrEN \\ }} \\[5mm]
  {\em Key words:} & \odstavec{\delka}{\keyword}
\end{tabular}
