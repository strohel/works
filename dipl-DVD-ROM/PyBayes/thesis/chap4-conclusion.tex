\clearpage % so that table of contents mentions correct page
\phantomsection % so that hyperref makes correct reference
\addcontentsline{toc}{chapter}{Conclusion}
\chapter*{Conclusion}
\vspace{-3mm} % HACK aby se conclusion veslo na jednu stranku
The theory of Bayesian filtering is introduced in the first chapter and the \emph{optimal
Bayesian solution} of the problem of recursive estimation is derived. Continues a survey of
well-known Bayesian filtering methods --- the Kalman filtering, particle filtering and the
marginalized particle filtering is described and properties of individual algorithms are discussed.

The second chapter contains a software analysis performed with the aim to identify the best
approach to software development and programming language for a desired library for Bayesian
filtering. Object-oriented approach is chosen along with the Python programming language, which is
found optimal except its potentially significant computational overhead. Cython is evaluated for the
task to improve Python performance with great success: a simple Python algorithm was 60\x\ faster
when compiled using Cython.

The last chapters presents the PyBayes library that was developed as a part of this thesis. PyBayes
builds on the software analysis performed in the previous chapter and is therefore object-oriented
and uses Python/Cython combination as its implementation environment and implements all presented
Bayesian filtering methods. To compare performance of Python/Cython combination in a real-world
example, the Kalman filter from PyBayes is benchmarked against MATLAB and C++ implementations from
BDM~\cite{BDM} with favourable results.

\noindent{}We believe that the \textbf{key contributions} of this thesis are:
\begin{itemize}
	\item The performed software analysis, that can be reused for a wide variety of software
		projects. In particular, we have shown that the choice of a high-level and convenient
		language such as Python is \emph{not necessarily} the enemy of speed. The analysis includes
		benchmarks with quite surprising results that show that Cython and PyPy are great speed
		boosters of Python.
	\item The PyBayes library itself. While it is not yet feature-complete, it provides a solid base
		for future development and is unique due to its dual-mode approach: it can be both treated
		as ordinary Python package with all the convenience it brings or compiled using Cython for
		performance gains.
\end{itemize}
\textbf{Future work} includes extending PyBayes with more filtering algorithms (non-liner Kalman
filter variants etc.) in the long term and fixing little inconveniences that currently exist in
PyBayes in the sort term; version 0.4 that would incorporate all future changes mentioned in the
third chapter is planned to be released within a few months. We are also looking forward to
incorporate emerging projects into our software analysis, for example the PyPy project looks very
promising.
